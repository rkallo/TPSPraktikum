\section{Patientenvorstellung}

Bei der Patientin handelt es sich um eine Frau, die seit etwa 2015 Beschwerden in der
rechten Schulter hat. Der Befund ist Tendinitis calcarea und durch konservative Therapiemaßnahmen
kam es zu keiner Verbesserung. Die Patientin hat weitere Erkrankungen. Darunter fallen
Diabetes mellitus, arterielle Hypertonie und Herzrythmusstörungen. Außerdem ist sie
allergisch auf Diclofenac, Cortison und Pflaster. Da es durch konservative Therapiemaßnahmen
zu keiner Verbesserung kam, wird eine Strahlentherapie verordnet. Dabei wird die Schulter
mit einer Fraktionsdosis von $\SI{0.5}{\gray}$ drei mal pro Woche bestrahlt. Es soll
insgesamt sechs Sitzungen geben und somit eine Gesamtdosis von $\SI{3}{\gray}$ bestrahlt werden.
