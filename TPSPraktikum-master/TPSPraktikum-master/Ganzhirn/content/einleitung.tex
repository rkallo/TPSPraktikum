\section{Einleitung}
\label{sec:Einleitung}

Bei vielen malignen Tumoren kann es zu Metastasen in anderen Organen kommen. Ist
diese Metastasierung schon weit fortgeschritten, kann durch eine Strahlentherapie
keine vollständige Heilung mehr erreicht werden. Eine solche Behandlung wird auch
Palliative Behandlung genannt. Das Ziel einer palliativen Behandlung ist es
die Lebensqualität des Patienten zu verbessern. In diesem Beispiel hat eine Patientin
mehrere Metastasen im Großhirn. Um eine gute lokale Kontrolle zu erreichen, wird bei dieser
palliativen Strahlentherapie das gesamte Großhirn als Zielvolumen gewählt.


%Tumorerkrankung mit einer Hirnmetastasierung hier Melanom (Hauttumor)
%Metastasen in Lunge Hirn Knochen
%
%Ganzhirndurchflutung = Gesamtes Großhirn
%
%Fortgeschrittene Metastasierung
%Palliative Behandlung im vlg zu Kurativ = Heilungserfolg ist das Ziel
%
%Palliativ = Heilung nicht in Aussicht. Nicht auf bestmögliche Dosisversorgung des PTVs.
%Gute Dosisverteilung aber auch dass durch Strahlentherapie nicht zu viele Nebenwirkungen erzeugt werden.
%Risikoorgane Schützen.
%
%Hier nicht viele Risikoorgane nur auf Augen achten.
%Nur Linsentrübung, aber diese tritt erst in Jahren auf (3-5). Trotzdem Linsen schützen.
