\section{Patientenvorstellung}

Bei der Patientin ist ein malignes Melanom diagnostiziert worden. Dieses Melanom
hat bereits mehrfach metastasiert, wobei es zu Metastasen in der Lunge, in den Knochen und
im Großhirn gekommen ist. Bei einer Untersuchung sind neue Metastasen im Großhirn entdeckt worden und
ein Verdacht auf eine Metastase in der linken Nebenniere. Zu weiteren Diagnosen gehören
arterielle Hypertonie und Fettstoffwechselstörung. Die Patientin ist $\SI{163}{\centi\meter}$
groß und wiegt $\SI{70}{\kilo\gram}$. Aufgrund der neuen Metastasen, wird eine perkutane
Radiotherapie des Ganzhirns durchgeführt und eine fraktionierte Radiotherapie der linken Nebenniere.
In diesem Fall wird der Bestrahlungsplan für die Bestrahlung des Ganzhirns erstellt. Bei
dieser Therapie wird mit einer Dosis von $\SI{1.8}{\gray}$ pro Sitzung in fünf Sitzungen
pro Woche bestrahlt. Die Gesamtdosis beträgt $\SI{45}{\gray}$.
