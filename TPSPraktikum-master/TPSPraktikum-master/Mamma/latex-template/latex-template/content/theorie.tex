\section{Einleitung}
\label{sec:Einleitung}
In diesem Beispiel wird ein duktales Karzinom bei der weiblichen Brust mit einer Strahlentherapie behandelt. Dieser Tumor stellt eine Vorstufe zur Krebserkrankung der Brustdrüse dar. \cite{Mamma}
Es soll eine adjuvante Radiotherapie der linken Mamma und des Tumorbettes auf die gesamte Mamma stattfinden. Das bedeutet, zuerst wird der Tumor entfernt und danach soll eine Nachbehandlung stattfinden.


\section{Patientenvorstellung}
\label{sec:Patientenvorstellung}
Bei der Patientin ist ein duktales Karzinom an der linken Mamma im oberen äußeren Quadranten diagnostiziert worden. Neben der indizierten adjuvanten kurativen Radiotherapie soll auch eine antihormonellen Therapie stattfinden. Dabei wurde sie über die möglichen Wirkungen und Nebenwirkungen der Strahlentherapie durch den Arzt aufgeklärt.
Es erfolgt ebenfalls eine Dosisaufsättigung im Bereich des Tumorbettes. Zur weiteren Behandlung soll eine atemgetriggerte Bestrahlung bei der linksseitigen Brust angewendet werden.
Zur weiteren Diagnosen gehören eine Amoxicillin-Allergie, eine Nerven- und Sehnenverletzung im rechten Handgelenk als auch eine Cholezystektomie und eine Nierenstein-OP.
Die Patientin ist $\SI{178}{\centi\meter}$ groß, wiegt $\SI{80}{\kilogram}$ und hat eine uneingeschränkte Armbeweglichkeit.
In der ersten Bestrahlungsserie wird eine Gesamtdosis von $\SI{50,4}{\gray}$ appliziert, welche in Fraktionen von $\SI{1,8}{\gray}$ in 5 Sitzungen pro Woche appliziert werden soll. In der zweiten Bestrahlungsserie wird mit einer Gesamtdosis von $\SI{59,4}{\gray}$ bzw $\SI{66,6}{\gray}$ das kleinere PTV bestrahlt bzw. es erfolgt eine Dosisaufsättigung im Tumorbett. Es soll erreicht werden, dass die beiden PTVs von der $\SI{95}{\percent}$ Isodosenlinie umschlossen werden und es soll auf Target-Mean dosiert werden.
