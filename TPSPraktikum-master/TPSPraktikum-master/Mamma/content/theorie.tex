\section{Einleitung}
\label{sec:Einleitung}
Die häufigste Krebserkrankung bei Frauen ist das Mammakarzinom. Eine Vorstufe von
dem Mammakarzinom ist das duktale Carcinoma in situ. Bei dieser Erkrankung kommt es zu einer Veränderung der
Zellen in der Mamma, diese veränderten Zellen sind allerdings noch nicht in der Lage in umliegendes Gewebes einzudringen.
Aus dieser Erkrankung entsteht mit hoher Wahrscheinlichkeit ein Mammakarzinom. \cite{Mamma} \\
In diesem Beispiel wird eine adjuvante Radiotherapie der linken Mamma durchgeführt. Dabei werden zunächst die veränderten
Zellen operativ entfernt und daraufhin wird die Radiotherapie durchgeführt um nicht sichtbare Tumorzellen zu behandeln \cite{adjuvant}.


\section{Patientenvorstellung}
\label{sec:Patientenvorstellung}
Bei der Patientin ist ein duktales Carcinoma in situ in dem oberen äußeren Quadranten der linken Mamma diagnostiziert worden.
Es ist bereits eine brusterhaltende Therapie erfolgt und nun wird eine adjuvante kurative Radiotherapie durchgeführt.
Die Strahlentherapie wird mit einem Atem-Gating durchgeführt, damit eine höhere Genauigkeit erreicht werden kann. Bei dem Atem-Gating wird nur in
bestimmten Atemphasen bestrahlt, damit sich das Zielvolumen bei Bestrahlung immer an der selben Stelle befindet.
Zusätzlich wird eine antihormonelle Therapie über fünf Jahre durchgeführt.
Die Patientin wurde über mögliche Wirkungen und Nebenwirkungen der Strahlentherapie durch den Arzt aufgeklärt.
Zur weiteren Diagnosen gehören eine Amoxicillin-Allergie, eine Nerven- und Sehnenverletzung im
rechten Handgelenk und eine Cholezystektomie und eine Nierenstein-OP.
Die Patientin ist $\SI{178}{\centi\meter}$ groß, wiegt $\SI{80}{\kilogram}$ und hat eine uneingeschränkte Armbeweglichkeit.
Die Bestrahlung wird mit der Shrinking-Field-Technik in zwei Bestrahlungsserien durchgeführt.
In der ersten Bestrahlungsserie wird eine Gesamtdosis von $\SI{50,4}{\gray}$ appliziert, welche in Fraktionen von $\SI{1,8}{\gray}$
in 5 Sitzungen pro Woche appliziert werden soll. Bei dieser Serie wird die gesamte linke Mamma bestrahlt.
In der zweiten Bestrahlungsserie wird mit einer Gesamtdosis von $\SI{9}{\gray}$ ein kleineres PTV bestrahlt.
Bei dieser Serie ist das Tumorbett das Zielvolumen. Damit wird bei der gesamten Therapie eine Dosis von $\SI{59,4}{\gray}$ appliziert.
Es soll erreicht werden, dass die beiden PTVs sicher von der $\SI{95}{\percent}$ Isodosenlinie umschlossen werden.
