\section{Bestrahlungsplanung}
\label{sec:Bestrahlungsplanung}
Die beiden PTVs sind in den CT-Daten bereits eingezeichnet, als PTV1 und PTV2.
Dabei ist PTV1 das größere Zielvolumen und PTV2 das kleinere. Das PTV2 wird als PTV-High bezeichnet und enthält nur das Tumorbett.
Das PTV1 wird als PTV-Intermediate bezeichnet und enthält die gesamte linke Mamma.
Die Kontur des Körpers und die Kontur von Risikoorganen werden in den vorliegenden CT-Bilder noch eingezeichnet.
Risikoorgane bei dieser Bestrahlung sind die Lunge, das Herz und das Rückenmark.
Bei diesen Organen muss darauf geachtet werden, dass die Organdosisgrenzwerte nicht überschritten werden.
Im ersten Bestrahlungsplan wird nur das PTV1 betrachtet. Für die Bestrahlung dieses Zielvolumens werden 4 Felder verwendet.
Die Gantry-Rotationen und die Gewichtungen für das erste PTV sind in der Tabelle \ref{tab:Felder1} dargestellt.

\begin{table}
	\centering
	\caption{Die Gantry-Rotation, Gewichtung und Feldgröße der beim ersten Bestrahlungsplan verwendeten Felder (PTV1).}
	\label{tab:Felder1}
	\begin{tabular}{c c c c}
		\toprule
		Feld & Gantry-Rotation & Gewichtung & Feldgröße\\
		\midrule
		$1$ & $130°$   & $1,3$ & $\num{10}$x$\num{18.7} \si{\centi\meter\squared}$ \\
		$2$ & $310°$  & $1.3$ & $\num{10.3}$x$\num{19.1} \si{\centi\meter\squared}$ \\
		$1.0$ & $130°$ & $0.2$ & $\num{10}$x$\num{18.7} \si{\centi\meter\squared}$ \\
		$2.0$ & $310°$ & $0.2$ & $\num{10.3}$x$\num{19.1} \si{\centi\meter\squared}$ \\
		\bottomrule
	\end{tabular}
\end{table}

Bei allen Feldern werden MLCs verwendet, die manuell an das PTV1 angepasst werden.
Es werden bei dieser Bestrahlung zwei mal die gleichen Felder verwendet, allerdings sind
die Einstellungen der Lamellen der MLCs unterschiedlich. Diese Methode wird angewendet um eine
angemessene Dosisverteilung in dem Zielvolumen zu erhalten.
Für den zweiten Bestrahlungsplan für das PTV2 sind die Gewichtungen und die Gantry-Rotationen in der Tabelle \ref{tab:Felder2} zu sehen.
Hierbei wurden auch 4 Felder benutzt und es wurden auch MLCs verwendet, die manuell an das PTV2 angepasst worden sind.
Auch bei diesem Plan ist einmal ein Feld doppelt verwendet worden. Die identischen Felder
unterscheiden sich dadurch, dass die MLCs unterschiedlich eingestellt worden sind.
Beide Bestrahlungspläne werden auf \enquote{$\SI{100}{\percent}$ target mean} normiert.

\begin{table}
	\centering
	\caption{Die Gantry-Rotation, Gewichtung und Feldgröße der beim ersten Bestrahlungsplan verwendeten Felder (PTV2).}
	\label{tab:Felder2}
	\begin{tabular}{c c c c}
		\toprule
		Feld & Gantry-Rotation & Gewichtung & Feldgröße\\
		\midrule
		$1$ & $140°$   & $0.65$ & $\num{5.9}$x$\num{6.3} \si{\centi\meter\squared}$ \\
		$2$ & $320°$  & $0.8$ & $\num{5.9}$x$\num{6.3} \si{\centi\meter\squared}$ \\
		$3$ & $55°$ & $0.3$ & $\num{5.9}$x$\num{6.3} \si{\centi\meter\squared}$ \\
		$1.0$ & $140°$ & $0.1$ & $\num{5.9}$x$\num{6.3} \si{\centi\meter\squared}$ \\
		\bottomrule
	\end{tabular}
\end{table}
