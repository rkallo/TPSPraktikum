\textbf{4. Erläutern Sie den Unterschied zwischen seriellen und parallelen Organen und
welche Bedeutung es für die Festlegung von Toleranzdosen (QUANTEC) hat.}

Organe werden in Untereinheiten aufgeteilt. Anhand dieser Aufteilung können Organe in
serielle oder parallele Organe aufgeteilt werden. Bei seriellen Organen sind die Untereinheiten
liegen die Untereinheiten in serieller Form vor.
Das bedeutet, wenn eine dieser Untereinheiten beschädigt wird, ist das gesamte Organ
nicht mehr im Stande seine Funktion zu erfüllen. Dazu zählt zum Beispiel das Rückenmark.

Bei parallelen Organen liegen diese Untereinheiten parallel vor. Dabei kann das Organ auch
bei Beschädigung von einer dieser Untereinheiten seine Funktion, im gewissen Rahmen, noch erfüllen.
Zu diesen Organen zählt zum Beispiel die Lunge.

Es gibt auch Organe, die sowohl seriell als auch parallele Untereinheiten haben, wie zum Beispiel das Herz.

Bei seriellen Organen hat Anteil des bestrahlten Volumens nur einen geringen Einfluss auf
die Toleranzdosis. Bei dem Rückenmark ist die Toleranzdosis zum Beispiel für die Dosis
pro $\SI{5}{\centi\meter}$, pro $\SI{10}{\centi\meter}$ und pro $\SI{20}{\centi\meter}$
sehr ähnlich.

Bei parallelen Organen hingegen, können kleine Volumenanteile hohen Dosen ausgesetzt
werden. Deshalb verringert sich die Toleranzdosis stark mit dem bestrahlten Volumen
des Organs. \\\\

\textbf{5. Überlegen Sie sich, welche Aufgaben in Eclipse mit dem ”Konturierungsmodul”
und welche mit dem Modul ”perkutane Bestrahlungsplanung” erledigt werden können.}

Mit dem Konturierungsmodul können in den CT-Bildern des Patienten der Tumor, also das Tumorvolumen und das Klinische Zielvolumen,
Risikoorgane und anatomische Strukturen eingezeichnet werden. Dadurch werden die CT-Bilder
des Patienten segmentiert.

Mit dem Modul der perkutanen Bestrahlungsplanung, wird der tatsächliche Bestrahlungsplan
erstellt. Der Plan wird so erstellt, dass das CTV die erforderliche therapeutische Dosis erhält und
die Toleranzwerte der Risikoorgane nicht überschritten werden. \\\\

\textbf{6. Erläutern Sie einen Berechnungsalgorithmus Ihrer Wahl. Den Pencil-Beam-Algorithmus:}

Dieser Algorithmus simuliert die Wechselwirkung von Photonen mit Materie, indem der
Photonenstrahl in Elementarstrahlen zerlegt wird. Deren Dosisdeposition wird dabei
separat berechnet. Bei diesem Algorithmus wird die Form und Gewichtung des Strahlungsfeldes
durch die strahlenformenden Komponenten des Linearbeschleunigers bestimmt. Dazu gehören
Blenden, Blöcke oder Lamellen des Kollimators.
Beim Eintritt in das Gewebe kommt es zu Aufweichung der Einzelstrahlen und zu einer
tropfenförmigen Dosisverteilung im Gewebe.
Die Dosisdeposition der Strahlen in tieferen Gewebeschichten wird vereinfacht. Dabei wird
die mittlere Dichte in einem Wegintervall verwendet um die Strahlschwächung zu berechnen.

Durch die Vereinfachungen, die in diesem Algorithmus gemacht werden kommt es zu Fehlerquellen.
Vor allem im Feldgrenzbereich kommt es zu Abweichungen. \\\\

\textbf{Wie können Sie das in Abbildung 1.1 dargestellte Volumen mit 3 Feldern
homogen bestrahlen.}

Man könnte das Volumen mit einem Feld von vorne und mit zwei Feldern von jeweils beiden Seiten Bestrahlen.
Durch diese Bestrahlung würden das Rektum geschont werden. 
