\section{Theoretische Grundlagen}
\label{sec:Theorie}

\textbf{1. Erläutern Sie den Ablauf einer Bestrahlungsplanung!}

Für die Erstellung einer Bestrahlungsplanung sollen als erstes die wichtigsten Informationen zu der Anatomie des Patienten gesammelt werden. Das ist vor allem sehr wichtig, damit der Tumor und eventuelle Metastasen möglichst zielgenau bestrahlt zu werden. Es soll vor allem auch vermieden werden gesunde Organe oder Gewebe zu bestrahlen.
Dafür kommt ein 4D-CT zum Einsatz. Der Vorteil bei diesem CT ist, dass hierbei mehrere CT-Aufnahmen pro Schnittebene aufgenommen werden und dabei den zeitlichen Verlauf während der Atmung abbilden. Das ist vor allem wichtig, weil im Abdomen- als auch im Thoraxbereich oft Verschiebungen durch die Atmung entstehen.
Sobald diese Daten gesammelt wurden, kann nun die Bestrahlungsplanung schrittweise geplant werden. Dafür gibt es verschiedenste Programme für die Bestrahlung, die benutzt werden, um Tumoren, Risikoorgane und anatomische Strukturen zu kennzeichnen. Dies wird durch durch den Strahlentherapeut durchgeführt und im Anschluss wird die Bestrahlungsplanung hauptsächlich durch den Medizinphysik-Experten geplant.
Der Arzt muss sich aber auch den Bestrahlungsplan anschauen, damit dieser an den Therapiesimulator weitergegeben werden kann.
Im nächsten Schritt werden die Einstellungen am Simulator überprüft und die Bestrahlungsfelddaten an den Beschleuniger weitergeschickt. In mehreren Fraktionen erfolgt die Behandlung des Patienten.

\textbf{2. Nennen Sie die wichtigsten Komponenten eines LINACs und erläutern Sie kurz deren Aufgaben!}

\emph{Elektronenkanone:} Die zur Beschleunigung benötigten Elektronen werden in einer Elektronenquelle erzeugt. Dies passiert in einem Bereich von $\SI{5,5}{\mega\watt}$. 

\emph{Modulator:} enthält als wichtigstes Bauteil die Quelle zur Hochfrequenzerzeugung sowie die Steuerelektronik und die elektrische Versorgung. Die erzeugte Energie, die für die gewünschte Elektronen- und Photonenstärke notwendig ist, wird dem Mikrowellensystem zugegeben.

\emph{Mikrowellensystem:} Als Hochfrequenzquelle erzeugt hierbei ein Klystron Mikrowellen. Die erzeugte Energie, die für die gewünschte Elektronen- und Photonenstärke notwendig ist, wird über ein Hohlwellenleitersystem von der Hochfrequenzquelle ins Beschleunigungsrohr transportiert. Dort bildet sich eine fortlaufende Welle aus, auf der die Elektronen "surfen" und weiter an den Beschleunigerkopf beschleunigt und weitergegeben werden.

\textit{Umlenkmagnetsystem:} Hier wird als $\SI{270}{\circ}$ - Ablenkung System ausgelegt und sorgt für die Umlenkung des Elektronenstrahls in den Strahlerkopf und anschließende Weiterleitung an das Target.  

\textit{Kollimatorsystem:} Um den feinen Elektronenstrahl für klinische Anwendungen nutzen zu können, muss dieser aufgeweitet werden. Hierbei werden Streufolien benötigt, die alle auf einem Drehteller fixiert sind. Bei der Erzeugung von Photonen werden die beschleunigten Elektronen auf einen Target gelenkt (Wolfram), wo sie abgebremst werden und dabei ultrahalte Bremsstrahlung erzeugen.

\textit{Steuerspulen:} Im Beschleunigerrohr kann der Elektronenstrahl auf Lage und Winkel überprüft werden.

\textit{Monitorsystem:} Dies besteht aus einer Anzahl von Ionisationskammer, die durchstrahlt werden und mit deren Hilfe sowohl die Dosis überwachen als auch die Homogenität des Strahlenfeldes laufend kontrollieren.  

\textit{Multilamellenkollimator:} Eine Vielzahl von Wolframscheiben, die paarweise einander gegenüberliegend angeordnet sind, können computergesteuert so verstellt werden, dass beliebig geformte Strahlfelder erzeugt werden können. Dies erlaubt eine Anpassung des Strahlquerschnitts an die Form des erkrankten Gewebe. 


\textbf{3. Definieren Sie die unterschiedlichen Bestrahlungsvolumina, die zur Findung des zu behandelnden Volumens (PTV) nötig sind.}

Bei der Behandlung maligner Erkrankungen werden folgende Bereiche unterschieden:

Das \textit{makroskopische Tumorvolumen} (GTV - Gross Tumor Volume) ist im Prinzip das durch Diagnostik maligne Gewebe. 

Bei dem \textit{klinischen Zielvolumen} (CTV - Clinical Target Volume) werden neben dem makroskopischen Tumorvolumen auch Bereiche mit einbezogen, in denen mögliche infiltrierte Tumorzellen vermutet werden.
Deshalb muss der zu behandelnde Bereich vergrößert werden.

Für das \textit{Planungs Zielvolumen} (Planning Target Volume - PTV) werden nicht nur räumliche Lageverschiebungen, z.B. durch Atmung oder Peristaltik berücksichtigt sondern auch beschränkte Genauigkeit in der Reproduzierung der Lagerung des Patienten. 

Das \textit{Behandelte Volumen} (Treated Volume - TV) bezeichnet das Volumen, welches die therapeutische Dosis enthält, die als ausreichend für das Erreichen des Behandlungszieles geschätzt wird. ($\SI{95}{\percent}$ der Herddosis)

Bei dem \textit{Bestrahlten Volumen} (Irradiated Volume - IV) handelt es sich um das Volumen, das in Bezug auf einer therapeutischen Bestrahlung einer signifikanten Dosis exponiert wird.

