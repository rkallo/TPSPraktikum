\section{Patientenvorstellung}
\label{sec:Durchführung}
Bei dem Patient ist ein Rektum Karzinom diagnostiziert worden.
Im Jahr 2014 wurden bei ihm rektale Blutungen festgestellt und aus diesem Grund ist eine
Rektoskopie und eine Koloskopie durchgeführt worden.
Der Patient wiegt $\SI{88}{\kilogram}$, ist $\SI{176}{\centi\meter}$ groß und es besteht keine Stuhl- oder Urininkontinenz.
Außerdem leidet der Patient an einer arteriellen Hypertonie und Prostatahyperplasie.
Es soll eine kurativ intendierte neoadjuvante kombinierte Chemo-Strahlentherapie stattfinden,
wobei eine Gesamtdosis von $\SI{50,4}{\gray}$ in
Shrinking-Field-Technik appliziert wird.
Bei der Shrinking-Field-Technik wird nach einer ersten Bestrahlungsserie das Zielvolumen in einer zweiten Serie verkleinert.
In der ersten Bestrahlungsserie wird eine Gesamtdosis von $\SI{45,4}{\gray}$ appliziert,
welche in Fraktionen von $\SI{1,8}{\gray}$ in 5 Sitzungen pro Woche appliziert werden soll.
In der zweiten Serie wird mit einer Gesamtdosis von $\SI{5,4}{\gray}$ das kleinere PTV bestrahlt,
welche auch in Fraktionen von $\SI{1,8}{\gray}$ in 5 Sitzungen pro Woche appliziert werden soll.
Es soll erreicht werden, dass das die beiden PTVs von der $\SI{95}{\percent}$ Isodosenlinie umschlossen werden.
Im Anschluss an die neoadjuvante Strahlentherapie wird der Tumor operativ entfernt.
