\section{Patientenvorstellung}
\label{sec:Durchführung}
Bei dem Patient ist ein Rektumkarzinom diagnostiziert worden. Im Jahr 2014 wurden bei ihm rektale Blutungen festgestellt und es wurden eine Reihe von Untersuchungen (Rektoskopie mit Probenentnahme und Koloskopie) durchgeführt. 
Der Patient wiegt $\SI{88}{\kilogram}$, ist $\SI{176}{\centi\meter}$ groß und bei ihm gibt es keine Stuhl- als auch Urininkontinenz. Außerdem leidet er an einer arteriellen Hypertonie und Prostatahyperplasie. Dabei wurde er über die möglichen Wirkungen und Nebenwirkungen der Strahlentherapie durch den Arzt aufgeklärt. 
Im weiteren Verlauf soll eine kurativ intendierte neoadjuvante kombinierte Chemo-Strahlentherapie stattfinden, wobei eine Gesamtdosis von $\SI{50,4}{\gray}$ in
Shrinking-Field-Technik appliziert werden soll. Bei der Shrinking-Field-Technik wird nach einer ersten Bestrahlungsserie das Zielvolumen in einer zweiten Serie verkleinert.
In der ersten Bestrahlungsserie wird eine Gesamtdosis von $\SI{45,4}{\gray}$ appliziert, welche in Fraktionen von $\SI{1,8}{\gray}$ in insgesamt 5 Sitzungen pro Woche appliziert werden soll.
In der zweiten Serie wird mit einer Gesamtdosis von $\SI{5,4}{\gray}$ das kleinere PTV bestrahlt, welche auch in Fraktionen von $\SI{1,8}{\gray}$ in insgesamt 5 Sitzungen pro Woche appliziert werden soll. Es ergibt sich also eine Gesamtzielvolumendosis von $\SI{50,4}{\gray}$.
Es soll erreicht werden, dass das die beiden PTVs von der $\SI{95}{\percent}$ Isodosenlinie umschlossen werden. 