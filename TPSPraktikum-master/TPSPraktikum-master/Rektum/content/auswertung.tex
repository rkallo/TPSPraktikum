\section{Bestrahlungsplanung}
\label{sec:Bestrahlungsplanung}
Bei dieser Strahlentherapie wird die Shrinking-Field-Technik angewendet und aus diesem Grund werden zwei Bestrahlungspläne benötigt.
Bei diesen CT-Bildern ist darauf zu achten, dass der Patient auf dem Bauch liegt. Das bedeutet, dass in diesem
Fall die anterior Richtung und die posterior Richtung vertauscht ist.
Die beiden PTVs sind in den CT-Daten bereits eingezeichnet, als PTV1 und PTV2. Dabei ist PTV1 das größere Zielvolumen und PTV2 das kleinere.
Die Kontur des Körpers und die Kontur von Risikoorganen werden in den
vorliegenden CT-Bilder noch eingezeichnet. Risikoorgane bei dieser Bestrahlung sind die Blase und die Hüftköpfe. Bei dieser Planung muss darauf geachtet
werden, dass der Organdosisgrenzwert nicht überschritten wird. Im ersten Bestrahlungsplan wird nur das PTV1 betrachtet.
Für die Bestrahlung dieses Zielvolumens (PTV1) werden fünf Felder verwendet.
Die Gantry-Rotationen und die Gewichtungen für das erste PTV sind in der Tabelle \ref{tab:Felder1} dargestellt.

\begin{table}
	\centering
	\caption{Die Gantry-Rotation, Gewichtung und Feldgröße des ersten Bestrahlungsplan verwendeten Feldern.}
	\label{tab:Felder1}
	\begin{tabular}{c c c c}
		\toprule
		Feld & Gantry-Rotation & Gewichtung & Feldgröße\\
		\midrule
		$1$ & $0°$   & $0,30$ & $\num{14.7}$x$\num{24.4} \si{\centi\meter\squared}$ \\
		$2$ & $90°$  & $0,10$ & $\num{15.7}$x$\num{25.2} \si{\centi\meter\squared}$ \\
		$3$ & $270°$ & $0,10$ & $\num{16.5}$x$\num{25.2} \si{\centi\meter\squared}$ \\
		$4$ & $130°$ & $0,25$ & $\num{16.5}$x$\num{26.9} \si{\centi\meter\squared}$ \\
		$5$ & $230°$ & $0,25$ & $\num{17.6}$x$\num{26.9} \si{\centi\meter\squared}$ \\
		\bottomrule
	\end{tabular}
\end{table}

Bei allen Feldern werden MLCs verwendet, die jeweils manuell an das PTV1 angepasst worden sind.
Bei den Einstellungen der Lamellen ist darauf geachtet worden, dass die Blase so gut wie möglich geschont wird.
Der Bestrahlungsplan wird auf "$\SI{100}{\percent}$ target mean"  normiert.
Für den Bestrahlungsplan für das PTV2 werden die gleichen Gantry-Rotationen und Gewichtungen verwendet, aber in diesem Fall werden die Feldgrößen angepasst.
Die Daten zu diesen Feldern sind in der Tabelle \ref{tab:Felder2} dargestellt.
Hierbei werden auch MLCs wieder verwendet, die jeweils manuell an das PTV2 angepasst werden.
Es wird hier auch darauf geachtet, dass die Blase und die Hüftkörper geschützt werden.
Dieser Bestrahlungsplan wird auch auf "$\SI{100}{\percent}$ target mean" normiert.

\begin{table}
	\centering
	\caption{Die Gantry-Rotation, Gewichtung und Feldgröße des zweiten Bestrahlungsplan verwendeten Feldern.}
	\label{tab:Felder2}
	\begin{tabular}{c c c c}
		\toprule
		Feld & Gantry-Rotation & Gewichtung & Feldgröße\\
		\midrule
		$1$ & $0°$ & $0,30$   & $\num{10.4}$x$\num{10} \si{\centi\meter\squared}$ \\
		$2$ & $90°$ & $0,10$  & $\num{8.6}$x$\num{10}\si{\centi\meter\squared}$ \\
		$3$ & $270°$ & $0,10$ & $\num{8.6}$x$\num{10}\si{\centi\meter\squared}$ \\
		$4$ & $130°$ & $0,25$ & $\num{8.6}$x$\num{10}\si{\centi\meter\squared}$ \\
		$5$ & $230°$ & $0,25$ & $\num{9.8}$x$\num{9.7} \si{\centi\meter\squared}$ \\
		\bottomrule
	\end{tabular}
\end{table}
