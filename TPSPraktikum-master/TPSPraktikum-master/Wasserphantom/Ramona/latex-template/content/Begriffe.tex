\section{Begriffserklärung}\label{Begriffe}

\textbf{Planning Target Volume (PTV):}

Vor der Bestrahlungsplanung muss zunächst von einem Arzt das Tumorvolumen (GTV)
in dem CT-Bild eingezeichnet werden. Da es allerdings sein kann, dass diagnostisch
nicht sichtbare Bereiche mit Tumorzellen infiltriert sein können wird das GTV zu
einem Klinischen Zielvolumen (CTV) erweitert. Bei der Lage des Tumors gibt es
allerdings weitere Unsicherheiten.
Zum einen Verformung oder Verschiebung des Tumors aufgrund von Atmung oder Herzschlag und
zum anderen Lagerungsunsicherheiten bei der Bestrahlung. Aus diesem Grund wird
das PTV definiert, welches diese Unsicherheiten möglichst mit berücksichtigt. \cite{grundlagen}\\

\textbf{CT-Wert:}

Bei der Aufnahme von CT-Bildern, wird die Abschwächung der Photonenstrahlung durch
die Hounsfield-Skala beschrieben. Der CT-Wert kann dabei aus den Schwächungskoeffizienten
des Gewebes $\mu_\text{Gewebe}$ und von Wasser $\mu_\text{Wasser}$ bestimmt werden.

\begin{equation*}
  \text{CT-Wert} = \frac{\mu_\text{Gewebe} - \mu_\text{Wasser}}{\mu_\text{Wasser}} \cdot 1000 \text{HU}
\end{equation*}

In der Praxis geht diese Skala von etwa $-1000$ HU bis $3000$ HU. In diesem Fall ist der
CT-Wert von Wasser $0$ HU, von Luft $-1000$ HU, von Fett $-100$ HU und Knochen etwa $1000$ HU. \cite{HU}\\

\textbf{Referenzpunkt:}

Der Referenzpunkt ist der Punkt an dem das Strahlenfeld die im Bestrahlungsplanungsprogramm
eingestellte Geometrie und Größe besitzt. Der Referenzpunkt wird dabei meistens in
das Zentrum des Zielvolumens gelegt. \cite{grundlagen}\\

\textbf{Multi-Leaf-Collimator (MLC):}

Der MLC ist eine Möglichkeit nicht rechteckige Strahlenfelder zu ermöglichen.
Dieser Besteht aus einzelnen Lamellen, wobei jede einzeln von einem eigenen Motor
gesteuert werden kann. Die Lamellen bestehen meistens aus Wolfram.
Die einzelnen Lamellen können bei der Bestrahlung so verstellt werden, dass das
Strahlenfeld an die Tumorform angepasst wird. \cite{grundlagen}\\

\textbf{Isodosenlinien und -flächen:}

Bei Isodosenlinien ist die Dosis entlang dieser Linie konstant. Bei Isodosenflächen
ist die Dosis bei einer dreidimensionalen Dosisverteilung auf diesen Flächen konstant. \cite{grundlagen}\\

\textbf{Querprofil und Tiefendosiskurven (TDK):}

Ein Querprofil umfasst die Dosisverteilung eines Strahlenfeldes senkrecht zum Zentralstrahl und entlang einer
der Feldachsen. Eine TDK hingegen beschreibt den Dosisverlauf entlang des Zentralstrahls. \cite{grundlagen}\\

\textbf{Dosis-Volumen-Histogramm (DVH):}

Bei einem Dosis-Volumen-Histogramm wird die relative Dosis, die ein bestimmtes
Strukturvolumen erhält, gegen das relative Volumen dieser Struktur
dargestellt. Dadurch wird die räumliche Dosisverteilung eines Volumens auf
eine zweidimensionale Darstellung reduziert. Durch diese Darstellung der
Dosisverteilung geht räumliche Information verloren, allerdings kann mit
einem DVH die geplante Dosisverteilung effektiv beurteilt werden.
DVHs werden in der Klinik dafür eingesetzt um zu überprüfen ob das
PTV die gewünschte Dosis erhält und ob die Risikoorgane hinreichend geschont
werden. \cite{grundlagen}\\

\textbf{Normierungspunkt:}

Der Normierungspunkt ist ein Punkt im Zielvolumen, der als Bezugspunkt für die
Dosisnormierung verwendet wird. \cite{grundlagen}

\newpage
