\section{Bestrahlungsplanung}
\label{sec:Durchführung}

Bevor mit der Bestrahlungsplanung begonnen wird, werden zunächst wichtige Strukturen in die
CT-Daten eingezeichnet. Dabei wird zum einen der gesamte Thorax konturiert und auch mögliche
Risikoorgane. Risikoorgane bei dieser Therapie sind die Lunge, das Herz, das Rückenmark und die
Speiseröhre. Bei diesen Organen muss darauf geachtet werden, dass die Organdosisgrenzwerte nicht
überschritten werden. Die beiden PTVs sind bereits in den CT-Daten eingezeichnet, als PTV1 und PTV2.
Dabei ist PTV1 das größere Zielvolumen der ersten Bestrahlungsserie und PTV2 das kleinere.
Da bei dieser Strahlentherapie mit der Shrinking-Field-Technik gearbeitet wird, müssen
zwei Bestrahlungspläne erstellt werden.
Bei der ersten Bestrahlungsserie wird nur das PTV1 betrachtet. Bei dieser Bestrahlung werden
drei Felder verwendet. Die Gewichtungen, Gantry-Rotationen und Größen dieser Felder sind in der
Tabelle \ref{tab:Felder1} dargestellt.

\begin{table}
  \centering
  \caption{Die Gantry-Rotation, Gewichtung und Feldgröße der bei dem ersten Bestrahlungsplan verwendeten Feldern.}
  \label{tab:Felder1}
  \begin{tabular}{c c c c}
    \toprule
    Feld & Gantry-Rotation & Gewichtung & Feldgröße\\
    \midrule
    $1$ & $235°$ & $0,8$ & $9,3$x$11,0 \si{\centi\meter\squared}$ \\
    $2$ & $120°$ & $1,0$ & $10,1$x$10,4 \si{\centi\meter\squared}$ \\
    $3$ & $320°$ & $1,0$ & $11,0$x$10,7 \si{\centi\meter\squared}$ \\
    \bottomrule
  \end{tabular}
\end{table}

Dabei sind bei allen Feldern MLCs verwendet worden. Die MLCs sind manuell an das PTV1 angepasst worden.
Bei dem zweiten Bestrahlungsplan wird das kleinere PTV2 betrachtet. Für diese Bestrahlung sind vier Felder
verwendet worden. Die Daten dieser Felder sind in der Tabelle \ref{tab:Felder2} gezeigt.

\begin{table}
  \centering
  \caption{Die Gantry-Rotation, Gewichtung und Feldgröße der bei dem zweiten Bestrahlungsplan verwendeten Feldern.}
  \label{tab:Felder2}
  \begin{tabular}{c c c c}
    \toprule
    Feld & Gantry-Rotation & Gewichtung & Feldgröße\\
    \midrule
    $1$ & $0°$   & $0,6$ & $9,8$x$7,6 \si{\centi\meter\squared}$ \\
    $2$ & $90°$  & $0,8$ & $9,4$x$7,6 \si{\centi\meter\squared}$ \\
    $3$ & $270°$ & $0,7$ & $9,4$x$7,6 \si{\centi\meter\squared}$ \\
    $4$ & $180°$ & $1,0$ & $9,8$x$7,6 \si{\centi\meter\squared}$ \\
    \bottomrule
  \end{tabular}
\end{table}

Auch bei diesen Feldern sind immer MLCs verwendet worden, die an das PTV2 angepasst worden sind.
Beide Bestrahlungspläne werden auf \enquote{$100\%$ target mean} normiert.
