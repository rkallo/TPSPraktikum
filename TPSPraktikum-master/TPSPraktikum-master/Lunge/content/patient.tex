\section{Patientenvorstellung}

Die Patientin leidet an persisterierenden Husten und bei einer CT-Untersuchung ist der Verdacht auf ein Lungenkarzinom aufgetreten.
Durch eine Bronchoskopie mit histologischer Diagnosesicherung ist ein lokal fortgeschrittenes und
lymphogen metastasierendes kleinzelliges Lungenkarzinom rechtsseitig diagnostiziert worden.
Aus diesem Grund ist eine potentiell kurativ intendierte kombinierte Chemo-Strahlentherapie
verordnet worden. Die Patientin ist $\SI{161}{\centi\meter}$ groß und wiegt $\SI{53}{\kilo\gram}$.
Durch eine Skelettszintigraphie ist festgestellt worden, dass keine Knochenmetastasen vorliegen.
Bei einem MRT des Schädels ist eine meningeale Metastase gefunden worden. Aus diesem Grund wird gegebenenfalls
anschließend eine prophylaktische Ganzhirnbestrahlung durchgeführt.
Zu weiteren Diagnosen zählen arterielle Hypertonie, Hyperurikämie, Polyarthrose in den Fingergelenken und eine Nickelallergie.
Außerdem ist die Patientin Raucherin. Die Strahlentherapie wird mit der Shrinking-Field-Technik in zwei Bestrahlungsserien durchgeführt.
Dabei wird bei der ersten Bestrahlungsserie ein größeres PTV bestrahlt und bei der zweiten Serie ein kleineres.
In der ersten Serie wird der Tumor und die mediastinalen Lymphabflusswege in das PTV aufgenommen. Im Anschluss
wird das PTV auf den sichtbaren Tumor und befallene Lymphknoten reduziert. Bei der ersten Serie wird eine Summendosis von
$\SI{50.4}{\gray}$ fünf mal pro Woche mit $\SI{1.8}{\gray}$ pro Sitzung appliziert. Bei der zweiten Serie wird eine Dosis von $\SI{9}{\gray}$
appliziert, dass sich eine Gesamtdosis von $\SI{59.4}{\gray}$ ergibt. Die beiden PTVs sollten sicher
von der $95\%$ Isodosenlinie umschlossen werden.
