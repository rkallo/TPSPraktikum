\section{Einleitung}
\label{sec:Einleitung}

Die Strahlentherapie kann bei verschiedenen Krankheitsbildern eingesetzt werden.
Dabei ist die bekannteste Anwendung bei Onkologischen Erkrankungen. Allerdings
kann sie auch bei Orthopädischen Erkrankungen zur Schmerzlinderung eingesetzt werden.
In diesem Beispiel geht es um so eine Orthopädischen Erkrankung.
Es soll mittels einer Strahlentherapie eine Schmerzlinderung bei Fersensporn erreicht werden.

\section{Patientenvorstellung}

Bei der Patientin ist Fersensporn an der linken Ferse diagnostiziert worden.
Die ersten Beschwerden traten im Jahr 2013 auf
und die konservativen Therapiemaßnahmen haben bisher keinen Erfolg gezeigt. Deshalb muss die Patientin zur Beschwerdelinderung
bestrahlt werden. Die Patientin befindet sich bei einer Voruntersuchung in einem guten Allgemeinzustand.
Sie wiegt $\SI{90}{\kilo\gram}$ und ist $\SI{159}{\centi\meter}$ groß.
Es wird eine Bestrahlung mit einer Gesamtdosis von $\SI{3}{\gray}$ verordnet, welche in Fraktionen von
$\SI{0.5}{\gray}$ appliziert werden soll in insgesamt 6 Bestrahlungen.
Dabei sollen 3 Bestrahlungen pro Woche stattfinden.
Dies soll helfen, damit der Körper genug Zeit hat, auf die Behandlung zu reagieren.
