\section{Bestrahlungsplanung}

Bevor der Bestrahlungsplan erstellt werden kann, müssen in den CT-Daten wichtige Strukturen eingezeichnet werden.
Das PTV und der gesamte Oberkörper ist bereits konturiert, deshalb müssen noch die Risikoorgane
eingezeichnet werden. Die Risikoorgane sind bei dieser Bestrahlung die Lunge, das Herz, die Nieren, die Leber und
das Rückenmark. Bei diesen Organen muss darauf geachtet werden, dass die Organdosisgrenzwerte nicht
überschritten werden. Für die Bestrahlung werden sechs Felder verwendet. Die Gewichtung, Gantry-Rotation
und Größe der Felder sind in der Tabelle \ref{tab:Felder} dargestellt.

\begin{table}
  \centering
  \caption{Die Gantry-Rotation, Gewichtung und Feldgröße der bei dem Bestrahlungsplan verwendeten Feldern.}
  \label{tab:Felder}
  \begin{tabular}{c c c c}
    \toprule
    Feld & Gantry-Rotation & Gewichtung & Feldgröße\\
    \midrule
    $1$   & $270°$ & $0,85$ & $11,5$x$17,3 \si{\centi\meter\squared}$ \\
    $2$   &  $90°$ & $0,85$ & $11,5$x$17,6 \si{\centi\meter\squared}$ \\
    $3$   & $180°$ & $0,80$ & $11,5$x$17,7 \si{\centi\meter\squared}$ \\
    $4$   &   $0°$ & $0,80$ & $11,5$x$17,7 \si{\centi\meter\squared}$ \\
    $1.0$ & $270°$ & $0,15$ & $11,5$x$17,3 \si{\centi\meter\squared}$ \\
    $2.0$ &  $90°$ & $0,15$ & $11,5$x$17,6 \si{\centi\meter\squared}$ \\
    \bottomrule
  \end{tabular}
\end{table}

Dabei werden bei allen Feldern MLCs verwendet, die an das PTV angepasst werden.
Es werden zwei mal die gleichen Felder verwendet, dabei werden allerdings die MLCs
unterschiedlich eingestellt. Diese Methode wird verwendet, das das PTV sehr groß ist und
sich die Umgebung des PTVs stark ändert und somit die Strahlenfelder unterschiedlich stark
abgeschwächt werden. Identische Felder mit unterschiedlichen MLC Einstellungen sorgen dafür, dass
sich die gewünschte Dosisverteilung in dem PTV einstellt.
Der Plan wird auf \enquote{$100\%$ target mean} normiert.
