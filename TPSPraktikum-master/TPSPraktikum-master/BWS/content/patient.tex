\section{Patientenvorstellung}

Bei der Patientin ist bereits im Jahr 1998 ein invasives ductales Mammacarcinom
diagnostiziert worden. In diesem Jahr ist eine adjuvante Radiotherapie
durchgeführt worden. In dem Jahr 2013 sind das erste mal ossäre Metastasen festgestellt worden
und diese sind auch bereits durch eine palliative Radiotherapie behandelt worden. Dabei sind
die BWK 1-4 und die LWK 1 - Os sacrum, sowie das Sternum mit $\SI{45}{\gray}$ bestrahlt worden.
Nun sind neue ossäre Metastasen festgestellt worden und die BWK 7-12 sollen entsprechend
mit einer palliativen Ratiotherapie behandelt werden. Bei dieser Bestrahlung müssen die
vorher durchgeführten Radiotherapien nicht berücksichtigt werden, da es sich um ein strahlentherapeutisch
nicht vorbelastetes Gebiet handelt.
Die Patientin befindet sich in einem schlechten Allgemeinzustand und hat Schmerzen beim Gehen.
Zu weiteren Diagnosen zählt intermittierendes Vorhofflimmern, Hypothyreose, Diabetes mellitus und
arterielle Hypertonie. Bei dieser Strahlentherapie werden die BWK 7-12 mit einer Gesamtdosis von
$\SI{45}{\gray}$ bestrahlt. Dabei werden fünf Bestrahlungen pro Woche mit einer Dosis von
$\SI{1.8}{\gray}$ pro Sitzung durchgeführt.
