\section{Patientenvorstellung}
\label{sec:Patientenvorstellung}
Seit Juni 2014 hat es sich bei dem Patient eine Schilddrüsenteilresektion festgestellt und hierzu erfolgte eine Substitutionstherapie. Anschließend hat es zu einer ausgedehnten endokrinen Orbitopathie entwickelt. Für eine Linderung der Symptome sind eine Reihe von Therapiemaßnahmen durchgeführt worden. Dazu gehört eine Kortisonstoßtherapie sowie Lymphdrainage. Diese Therapiemaßnahmen haben nicht geholfen und deshalb wird der Patient eine Strahlentherapie verordnet. Dabei wurde er über die möglichen Nebenwirkungen und Wirkungen der Strahlentherapie durch den Arzt aufgeklärt. Zur weiteren Diagnosen gehören vor allem eine Nephrektomie, die sich auf der linken Seite befindet und im Kinderalter festgestellt wurde. Außerdem ist der Patient ein Raucher und leidet an einer arteriellen Hypertonie. Der Patient ist $\SI{190}{\centi\meter}$ groß und wiegt $\SI{98}{\kilogram}$. Außerdem ist der Patient lichtempfindlich und durch die Erkrankung hat er ein Fremdkörpergefühl und tränt oft. Für die Bestrahlug wird eine Gesamtdosis von $\SI{19,8}{\gray}$ verschrieben, welche in Fraktionen von $\SI{1,8}{\gray}$ in insgesamt 5 Sitzungen appliziert werden soll. Insgesamt sollen 5 Bestrahlungen pro Woche stattfinden. Das ist hilfreich, denn der Körper muss genug Zeit haben, auf die Behandlung zu reagieren. Wichtig ist, dass die $\SI{95}{\percent}$ Isodosenlinie das PTV umschließt.