\section{Patientenvorstellung}
\label{sec:Patientenvorstellung}

Bei der Patient ist Morbus Basedow diagnostiziert worden.
Im Folge dessen wurde im Juni 2014 eine Schilddrüsenteilresektion durchgeführt und seitdem wird eine Substitutionstherapie durchgeführt.
Daraufhin hat sich eine ausgedehnte endokrine Orbitopathie entwickelt.
Für eine Linderung der Symptome sind eine Reihe von Therapiemaßnahmen durchgeführt worden. Dazu gehört eine Kortisonstoßtherapie sowie Lymphdrainage.
Diese Therapiemaßnahmen haben nicht geholfen und deshalb wird dem Patient nun eine Strahlentherapie verordnet.
Dabei wurde er über die möglichen Nebenwirkungen und Wirkungen der Strahlentherapie durch den Arzt aufgeklärt.
Der Patient ist Raucher und leidet an einer arteriellen Hypertonie. Der Patient ist $\SI{190}{\centi\meter}$ groß und wiegt $\SI{98}{\kilogram}$. Außerdem ist der Patient lichtempfindlich und durch die Erkrankung hat er ein Fremdkörpergefühl und tränt oft.
Für die Bestrahlung wird eine Gesamtdosis von $\SI{19,8}{\gray}$ verschrieben, welche in Fraktionen von $\SI{1,8}{\gray}$ in insgesamt elf Sitzungen
appliziert werden soll. Es sollen fünf Bestrahlungen pro Woche stattfinden. Das ist hilfreich, denn der Körper muss genug Zeit haben,
auf die Behandlung zu reagieren. Es soll erreicht werden, dass die $\SI{95}{\percent}$ Isodosenlinie das PTV umschließt.
