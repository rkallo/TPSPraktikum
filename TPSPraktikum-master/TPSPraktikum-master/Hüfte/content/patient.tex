\section{Patientenvorstellung}

Seit Mai 2015 hat die Patienten Beschwerden an der linken Hüfte. Zur Linderung der Symptome ist eine Spritzentherapie durchgeführt worden
und es wurden ihr NSAR verabreicht.
Diese konservativen Therapiemaßnahmen haben keinen Erfolg gezeigt. Aus diesem Grund soll die Patientin zur Linderung der Symptome bestrahlt werden.
Außerdem wurden ihr die möglichen Wirkungen und Nebenwirkungen der Strahlentherapie erklärt. Zu weiteren Diagnosen gehören eine
Schilddrüsenunterfunktion und Hypercholesterinämie. Sie ist $\SI{160}{\centi\meter}$ groß und hat ein Gewicht von $\SI{60}{\kilo\gram}$.
Die lateralen Druckschmerzen befinden sich an der linken Hüfte. Bei der Strahlentherapie soll eine Gesamtdosis von $\SI{3}{\gray}$ appliziert werden.
Die Dosis wird in Fraktionen von $\SI{0,5}{\gray}$ in insgesamt 6 Sitzungen appliziert und es werden 3 Bestrahlungen pro Woche stattfinden.
Das soll dabei helfen, dass der Körper genug Zeit hat, auf die Behandlung zu reagieren. Falls es nach 8-10 Wochen keine Besserung der Symptome gibt,
muss die Bestrahlung wiederholt werden.
