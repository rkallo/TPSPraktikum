\section{Patientenvorstellung}
Seit Oktober 2014 hat sich bei dem Patient eine rechts betonte Paraparese festgestellt und er hat eine weitere Diagnostik abgelehnt. Im Dezember 2014 wurde ein MRT bei ihm durchgeführt und es wurde ein fortgeschrittenes Stadium des Tumors festgestellt. Letztendlich wurde Anfang 2015 eine Probe entnommen und die Diagnose wurde sichergestellt. Erst April 2015 hat er sich in der Strahlentherapie mit der Tumorerkrankung im Stadium IV vorgestellt. Vor der Vorstellung wurde bei dem Patient der linke oberen Sprunggelenk und seine linke Ferse bestrahlt. Die Gesamtzielvolumendosis trug $\SI{3}{\gray}$. Außerdem ist der Patient an Diabetes mellitus Typ II und an arterieller Hypertonie erkrankt. Er ist 65 Jahre alt, wiegt $\SI{86}{\kilogram}$ und ist $\SI{172}{\centi\meter}$ groß. Außerdem sind es bei ihm andere neurologische Auffälligkeiten nicht vorhanden. Dabei wurde er über die möglichen Wirkungen und Nebenwirkungen der Strahlentherapie durch den Arzt aufgeklärt. Im weiteren Verlauf soll eine palliative perkutane fraktionierte Radiotherapie stattfinden und ihm wird eine Gesamtdosis von $\SI{50,4}{\gray}$ in der ersten Serie verschrieben, welche in Fraktionen von $\SI{1,8}{\gray}$ in insgesamt 5 Sitzungen pro Woche appliziert werden soll. Das ist hilfreich, denn der Körper muss genug Zeit haben, auf die Behandlung zu reagieren. In der zweiten Serie wird ihm eine Summendosis von $\SI{9}{\gray}$ verschrieben, welche auch in Fraktionen von $\SI{1,8}{\gray}$ in insgesamt 5 Sitzungen pro Woche appliziert werden soll. Es ergibt sich also eine Gesamtzielvolumendosis von  $\SI{59,4}{\gray}$. Wichtig ist, dass das PTV von der $\SI{95}{\percent}$ Isodosenlinie umschlossen wird. 