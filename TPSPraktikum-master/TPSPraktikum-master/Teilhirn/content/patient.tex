\section{Patientenvorstellung}
Im Oktober 2014 ist bei dem Patienten eine rechts betonte Paraparese festgestellt worden, aber der Patient hat eine weitere Diagnostik abgelehnt.
Im Dezember 2014 wurde ein MRT bei ihm durchgeführt und es wurde ein fortgeschrittenes Stadium des Tumors festgestellt.
Letztendlich wurde Anfang 2015 eine Probe entnommen und die Diagnose wurde sichergestellt.
Anfang April 2015 hat sich der Patient in der Strahlentherapie mit der Diagnose Glioblastoma multiforme vorgestellt.
Vor der Vorstellung wurde bei dem Patient das linke obere Sprunggelenk und die linke Ferse bestrahlt.
Die Gesamtzielvolumendosis betrug $\SI{3}{\gray}$. Außerdem ist der Patient an Diabetes mellitus Typ II und an arterieller Hypertonie erkrankt.
Der Patient ist 65 Jahre alt, wiegt $\SI{86}{\kilogram}$ und ist $\SI{172}{\centi\meter}$ groß.
Außerdem sind bei ihm andere neurologische Auffälligkeiten nicht vorhanden.
Er ist über die möglichen Wirkungen und Nebenwirkungen der Strahlentherapie durch den Arzt aufgeklärt worden.
Im weiteren Verlauf soll eine palliative perkutane fraktionierte Radiotherapie stattfinden, wobei eine Gesamtdosis von $\SI{59,4}{\gray}$ in
Shrinking-Field-Technik appliziert werden soll. Bei der Shrinking-Field-Technik wird nach einer ersten Bestrahlungsserie das Zielvolumen in
einer zweiten Serie verkleinert. Dabei wird bei der ersten Bestrahlungsserie die Tumorregion mit einem großem Sicherheitssaum bestrahlt, da es
sich bei diesem Hirntumor um einen sehr aggressiven Tumor handelt. In der zweiten Bestrahlungsserie wird der Sicherheitssaum verkleinert, um das
umliegende Gewebe zu schonen.
In der ersten Bestrahlungsserie wird eine Gesamtdosis von $\SI{50,4}{\gray}$ appliziert, welche in Fraktionen von $\SI{1,8}{\gray}$ in
insgesamt 5 Sitzungen pro Woche appliziert werden soll.
In der zweiten Serie wird mit einer Gesamtdosis von $\SI{9}{\gray}$ das kleinere PTV bestrahlt, welche auch in Fraktionen von $\SI{1,8}{\gray}$ in insgesamt
5 Sitzungen pro Woche appliziert werden soll. Es ergibt sich also eine Gesamtzielvolumendosis von  $\SI{59,4}{\gray}$.
Es soll erreicht werden, dass das die beiden PTVs von der $\SI{95}{\percent}$ Isodosenlinie umschlossen werden.
