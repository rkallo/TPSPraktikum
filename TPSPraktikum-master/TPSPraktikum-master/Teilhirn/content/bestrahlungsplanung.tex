\section{Bestrahlungsplanung}
\label{sec:Bestrahlungsplanung}

Da bei dieser Strahlentherapie die Shrinking-Field-Technik angewendet wird, werden zwei
Bestrahlungspläne benötigt.
Bevor die Bestrahlungspläne erstellt werden, muss die Kontur des gesamten Kopfes und von
Risikoorganen in die vorliegenden CT-Bilder eingezeichnet werden. Risikoorgane bei dieser
Therapie sind zum einen die Augenlinsen und zum anderen das Chiasma Optikum. Bei diesen
Organen muss drauf geachtet werden, dass der Organdosisgrenzwert nicht überschritten wird.
Die Zielvolumina der beiden Bestrahlungen sind bereits als PTV1 und PTV2 eingezeichnet. Dabei
ist PTV1 das größere Zielvolumen und PTV2 das kleinere.
In dem ersten Bestrahlungsplan wird nur das PTV1 betrachtet. Für die Bestrahlung dieses Zielvolumens werden
fünf Felder verwendet. Die Gantry-Rotationen und die Gewichtungen sind in der Tabelle \ref{tab:Felder1} dargestellt.

\begin{table}
  \centering
  \caption{Die Gantry-Rotation, Gewichtung und Feldgröße der bei beiden Bestrahlungsplänen verwendeten Feldern.}
  \label{tab:Felder1}
  \begin{tabular}{c c c c}
    \toprule
    Feld & Gantry-Rotation & Gewichtung & Feldgröße\\
    \midrule
    $1$ & $0°$ & $0,25$ & $10$x$10 \si{\centi\meter\squared}$ \\
    $2$ & $90°$ & $0,30$ & $15$x$15 \si{\centi\meter\squared}$ \\
    $3$ & $90°$ & $0,05$ & $15$x$15 \si{\centi\meter\squared}$ \\
    $4$ & $230°$ & $0,35$ & $15$x$15 \si{\centi\meter\squared}$ \\
    $5$ & $230°$ & $0,05$ & $15$x$15 \si{\centi\meter\squared}$ \\
    \bottomrule
  \end{tabular}
\end{table}

Bei allen Feldern werden MLCs verwendet, die jeweils an das PTV1 angepasst worden sind.
Bei der Bestrahlung werden Felder mit identischer Gantry-Rotation verwendet (Felder 2 und 3 und Felder 4 und 5), allerdings
sind bei den Feldern die MLCs unterschiedlich eingestellt worden um eine angemessene Dosisverteilung zu erhalten.
Bei den Einstellungen der Lamellen ist auch darauf geachtet worden, dass die Augenlinsen
so gut wie möglich geschont werden. Bei diesem Zielvolumen ist es nicht möglich das Chiasma
zu schützen, da es innerhalb des Zielvolumens liegt. \\
Für den Bestrahlungsplan für das PTV2 werden die gleichen Felder mit den gleichen
Gewichtungen wie bei dem ersten Plan verwendet. Diese Felder sind in Tabelle \ref{tab:Felder1} beschrieben.
Bei diesem Plan werden allerdings die MLCs an das neue Zielvolumen angepasst. Auch bei diesem
Plan sind mehrere Felder mit der gleichen Gantry-Rotation verwendet worden mit unterschiedlichen MLC Einstellungen um
eine gute Dosisverteilung zu erreichen. Außerdem
wird bei den Einstellungen der MLCs darauf geachtet, dass zum einen die Augenlinsen und
zum anderen das Chiasma ausreichend geschützt wird. Beide Pläne werden auf \enquote{$100\%$ target mean} normiert.
