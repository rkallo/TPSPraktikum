\section{Rechenaufgaben}

\subsection*{1. Aufgabe}
In dieser Aufgabe ist ein $7$ x $7\si{\centi\meter\squared}$ Strahlungsfeld
gegeben mit einer HVL von $\SI{1}{\milli\meter}$ Kupfer und einem BSF von
$\num{1.282}$.
Damit ergibt sich die Rückstreuung in Prozent zu

\begin{equation*}
  \%BSF = 100 \cdot (BSF - 1) = 28,2 \%.
\end{equation*}

Um die Dosis zu bestimmen, die an der Oberfläche des Patienten absorbiert
wird, wird aus der Tabelle $7.1$ die Prozentuale Tiefendose entnommen.
In dieser Tabelle sind die Werte für eine HVL von $\SI{2}{\milli\meter}$
Kupfer gegeben.
Da bei $d=\SI{0}{\centi\meter}$ $\%D_n = 100\%$ ist, wird bei einer
geringeren Photonenenergie, also bei $HVL = \SI{1}{\milli\meter}$ Kupfer,
dort der Wert auch $\%D_n = 100\%$ sein. Über die Formel für
die absorbierte Dosis $D_A$

\begin{equation*}
  D_A = R \cdot BSF \cdot \left(\frac{\%D_n}{100}\right) \cdot f
\end{equation*}

lässt sich dann die absorbierte Dosis an der Oberfläche bestimmen. Dabei
ist $R=105$ und $f = \SI{0.957}{\centi\gray\per\R}$. Damit ergibt sich die
absorbierte Dosis zu

\begin{equation*}
  D_A = \SI{128.82}{\centi\gray}.
\end{equation*}

\subsection*{2. Aufgabe}

In dieser Aufgabe ist ein Photonenfeld von $10$ x $10
\si{\centi\meter\squared}$, das aus $\SI{6}{\mega\volt}$ Photonen besteht.
Die $SSD = \SI{100}{\centi\meter}$ und es wird eine Dosis von
$\SI{1}{\centi\gray\per\MU}$ an dem Dosismaximum $d_m$ deponiert. Nun
soll bestimmt werden, wie viel MU eingestellt werden muss um
$\SI{200}{\centi\gray}$ in $\SI{8}{\centi\meter}$ Tiefe anzubringen.
Dazu muss zunächst bestimmt werden, wie viel Dosis pro Behandlung in
$\SI{8}{\centi\meter}$ Tiefe angebracht wird.

\begin{equation*}
  D_d = \SI{1}{\centi\gray\per\MU} \cdot \frac{\%D_n}{100} = \SI{0.745}{\centi\gray\per\MU}.
\end{equation*}

Dabei ist $\%D_n = 74,5 \%$ in $\SI{8}{\centi\meter}$ Tiefe aus der
Tabelle $7.3$
entnommen worden.
Die Behandlungszeit ergibt sich damit zu

\begin{equation*}
  t = \frac{\SI{200}{\centi\gray}}{\SI{0.745}{\centi\gray\per\MU}} = \SI{268}{\MU}.
\end{equation*}

Die Dosis pro Behandlung an dem Dosismaximum kann dann aus der Behandlungszeit
berechnet werden.

\begin{equation*}
  D_n = \SI{268}{\MU} \cdot \SI{1}{\centi\gray\per\MU} = \SI{268}{\centi\gray}
\end{equation*}

Für die Dosis an der Austrittsfläche in $\SI{20}{\centi\meter}$ Tiefe,
wird wieder aus der Tabelle $7.3$ $\%D_n = 38\%$ entnommen.
Damit ergibt sich die Dosis, die dort noch angebracht wird

\begin{equation*}
    D_d = \SI{1}{\centi\gray\per\MU} \cdot \frac{\%D_n}{100} = \SI{0.38}{\centi\gray\per\MU}.
\end{equation*}

und die Dosis ergibt sich damit zu

\begin{equation*}
  D_n = \SI{268}{\MU} \cdot \SI{0.38}{\centi\gray\per\MU} = \SI{101,84}{\centi\gray}
\end{equation*}

\subsection*{3. Aufgabe}

In dieser Aufgabe ist ein $\SI{26}{\centi\meter}$ dicker Patient gegeben, der
mit einem $8$ x $14 \si{\centi\meter\squared}$ Photonenfeld der Energie
$\SI{10}{\mega\volt}$ bestrahlt wird mit $SSD = \SI{100}{\centi\meter}$.
Durch dieses Feld wird eine Dosis von $\SI{200}{\centi\gray}$ in einer Tiefe
von \SI{10}{\centi\meter} deponiert. Um die Dosis an der Austrittsfläche zu
bestimmen wird zunächst mit der Tabelle $7.6$ ein äquivalentes quadratisches
Feld ermittelt. Das ist $10,1$ x $10,1 \si{\centi\meter\squared}$,
näherungsweise wird für die weitere Rechnung ein $10$ x $10
\si{\centi\meter\squared}$ Feld verwendet. Aus der Tabelle $7.4$ wird
$\%D_n=73\%$ in $\SI{10}{\centi\meter}$ Tiefe entnommen. Damit wird die
deponierte Dosis am Maximum bestimmt.

\begin{equation*}
  D_0 = \frac{D_n}{\%D_n} 100 = \SI{273.97}{\centi\gray}
\end{equation*}

Diese Formel wird umgestellt um die deponierte Dosis in
$\SI{26}{\centi\meter}$ Tiefe zu bestimmen. Der $\%D_n=34,8\%$ in dieser Tiefe.

\begin{equation*}
  D_n = \frac{\%D_n \cdot D_0}{100} = \SI{95,34}{\centi\gray}
\end{equation*}

\subsection*{4. Aufgabe}

In dieser Aufgabe ist ein $4$ x $4 \si{\centi\meter\squared}$ Feld
gegeben mit Photonen mit einer HVL von $\SI{2}{\milli\meter}$ Kupfer.
$SSD=\SI{50}{\centi\meter}$. Nun wird der SSD von $\SI{50}{\centi\meter}$
auf $\SI{60}{\centi\meter}$ verlängert. Um zu bestimmen, wie die MU
Einstellungen verändert werden müssen um die gleiche Dosis an der Oberfläche
zu erhalten wird das Abstandsquadrat Gesetz verwendet. Das besagt, dass
$D_r \sim \frac{1}{r^2}$. Damit werden die Dosen bei den verschiedenen
SSDs verglichen.

\begin{equation*}
  \frac{(D_n)_{60}}{(D_n)_{50}} = \left(\frac{\SI{50}{\centi\meter}}{\SI{60}{\centi\meter}} \right)^2 = \num{0.694}
\end{equation*}

Das bedeutet, dass durch die Änderung nurnoch $69,4\%$ der ursprünglichen
Dosis ankommt. Also müssen $30,6\%$ mehr MU eingestellt werden.
Nun wird noch die Änderung von $\%D_n$ abgeschätzt in $\SI{2}{\centi\meter}$
Tiefe.

\begin{equation*}
  \frac{(D_n)_{60}}{(D_n)_{50}} = \frac{\frac{(60+0)^2}{(60+2)^2}\exp(-2\mu)100}{\frac{(50+0)^2}{(50+2)^2}\exp(-2\mu)100} = 1,013
\end{equation*}

Das bedeutet $\%D_n$ steigt um $1,3\%$.

\subsection*{5. Aufgabe}

Ein Patient ist $\SI{22}{\centi\meter}$ tief und $\SI{32}{\centi\meter}$
breit. Es werden vier $10$ x $10 \si{\centi\meter\squared}$ Felder
verwendet, die jeweils senkrecht zueinander stehen.
$SSD=\SI{100}{\centi\meter}$ und $\SI{6}{\mega\volt}$ Photonen. In der
Mitte des Patienten, wo sich die Achsen der vier Felder treffen, soll eine
Dosis von $\SI{200}{\centi\gray}$ angebracht werden. Dabei soll jedes Feld
eine gleiche Dosis in diesem Punkt anbringen, also jedes Feld
$\SI{50}{\centi\gray}$. Für zwei Felder liegt dieser Punkt in
$\SI{11}{\centi\meter}$ Tiefe und für die anderen beiden in
$\SI{16}{\centi\meter}$ Tiefe. Die $\%D_n$ Werte werden aus der Tabelle $7.3$
entnommen und ergeben sich zu $\%D_{11} = 63,7\%$ und $\%D_{16} = 48,8\%$.
Mit der Gleichung

\begin{equation*}
  D_0 = \frac{D_n}{\%D_n} 100
\end{equation*}

lässt sich das Dosismaximum der einzelnen Felder bestimmen. Dabei muss
$D_n = \SI{50}{\centi\gray}$ sein.
Somit ergibt sich für die anterior-posterior Felder

\begin{equation*}
  D_0 = \SI{78.49}{\centi\gray}
\end{equation*}

und für die lateralen Felder

\begin{equation*}
  D_0 = \SI{102.46}{\centi\gray}
\end{equation*}

\subsection*{6. Aufgabe}

In dieser Aufgabe geht es um das gleiche Problem wie in Aufgabe 5. Nur diesmal
sollen alle Felder die gleiche maximale Dosis haben und damit
unterschiedliche Beiträge zu der gewünschten Dosis von $\SI{200}{\centi\gray}$
im Zentrum. Mit der Formel

\begin{equation*}
  D_n = \frac{\%D_n \cdot D_0}{100}
\end{equation*}

ergibt sich:

\begin{align*}
  2 D_{11} &+ 2 D_{16} &= \SI{200}{\centi\gray} \\
  2 \cdot 0,637 D_0 &+ 2 \cdot 0,488 D_0 &= \SI{200}{\centi\gray} \\
   & D_0 &= \SI{88.89}{\centi\gray}
\end{align*}

Damit ergibt sich für die anterior-posterior Felder eine Dosis im Zentrum von

\begin{equation*}
  D_n = 0,637 \cdot \SI{88.89}{\centi\gray} = \SI{56.62}{\centi\gray}
\end{equation*}

und für die lateralen Felder

\begin{equation*}
  D_n = 0,488 \cdot \SI{88.89}{\centi\gray} = \SI{43.38}{\centi\gray}.
\end{equation*}

\subsection*{7. Aufgabe}

In dieser Aufgabe wird ein Tumor, der unterhalb der Rippen liegt, mit einem
Elektronenstrahl bestrahlt. In der Strahlrichtung liegt $\SI{2}{\centi\meter}$
Knochen und $\SI{2}{\centi\meter}$ Gewebe. Die effektive Tiefe lässt sich
mit der Formel

\begin{equation*}
  d_{eff} = d - z(1-\text{CET})
\end{equation*}

bestimmen. Dabei ist $z = \SI{2}{\centi\meter}$, die Tiefe des dichtesten
Materials und CET kann als $1,65$ angenommen werden (vgl. S.179). Damit
ergibt sich die effektive Tiefe zu

\begin{equation*}
  d_{eff} = \SI{4}{\centi\meter} - \SI{2}{\centi\meter} (1 - 1,65) = \SI{5.3}{\centi\meter}
\end{equation*}
