\section{Theorie Aufgaben}
\label{sec:Theorie}

\textbf{1. Erläutern Sie drei Bestrahlungsparameter durch deren Einfluss $\si{\percent}$$D_n$ ansteigt.}


$\si{\percent}$$D_n$ variiert mit der Tiefe, Feldgröße, SSD und Energie des Strahls.
Zu den drei Bestrahlungsparameter gehören die Erhöhung der Feldgröße, die Vergrößerung der SSD und die Vergrößerung der Photonenenergie.

\textbf{
2. Erläutern Sie die Begriffe TAR (Tissue-air ratio) und BSF (Backscattering factor) und deren Abhängigkeit vom SSD.}

TAR ist definiert als:
\begin{equation*}
\text{TAR} = \frac{D_d}{D_\text{air}}
\end{equation*}
wobei $D_d$ die absorbierte Dosis an einem Ort im Medium und $D_\text{air}$ die absorbierte Dosis für kleine Gewebemasse, die an derselben Stelle in Luft umgeben ist. Manchmal wird es auch als Dosis im freien Raum $D_\text{fs}$ definiert. Normalerweise wird TAR auf der Zentralachse bestimmt. Außerdem ist TAR unabhängig vom SSD, weil die beiden Dosis $D_d$ und $D_\text{air}$ im selben Abstand von der Quelle gemessen werden.

BSF ist definiert als:
\begin{equation*}
\text{BSF} = \frac{D_0}{D_\text{air}}
\end{equation*}
wobei $D_0$ = $D_d$ ist. Es ist im Prinzip TAR, aber in der Tiefe des Dosismaximums $d_m$. Die Photonenstreuung, die in der Tiefe der Maximumdosis $d_m$ stattfindet, ist abhängig von der Menge des darunterliegenden Mediums, Größe, Form und Qualität der Gamma- oder Röntgenstrahlung. Genauso wie TAR, ist BSF auch unabhängig vom SSD aus demselben Grund wie TAR. Die Dosis $D_d$ und $D_\text{air}$ werden im selben Abstand von der Quelle gemessen.

\textbf{3. Definieren Sie den Begriff TPR (Tissue-Phantom Ratio).}

Es wird definiert als das Verhältnis einer Zentralachsendosis $D_n$ in einer Tiefe $n$ in einem Phantom zu einer Dosis an der gleichen Position, aber in
einer anderen Tiefe $t$ im Phantom. Diese Tiefe $t$ wird Referenztiefe genannt und ist üblicherweise $\SI{10}{\centi\meter}$.

\textbf{4. Wie können Tiefendosiskurven, die für die Bestrahlungsplanung von nöten sind, ermittelt werden? Warum werden diese benötigt?}

Tiefendosiskurven werden in der Regel einem Wasserphantom gemessen, da Wasser in Bezug auf Strahlungsstreuung und -absorption dem menschlichen Gewebe sehr ähnlich ist. Die Messungen erfolgen in der Regel mit einer Ionisationskammer. Die Tiefendosiskurven werden sowohl zur Kalibrierung des Dosiswertes an einer Bestrahlungsanlage, als auch zur Prüfung und Kalibrierung des Therapie Plansystems benötigt. Diese werden ermittelt, indem die relative Dosis entlang des Zentralstrahls in dem Wasserphantom gemessen wird.

\textbf{5. Nennen Sie verschiedene Gegenstände (mindestens 3!) mit denen Ungleichmäßigen in der Hautoberfläche ausgeglichen werden können.}

a) Bolus: Es produziert eine glattere Oberfläche über ein fast homogenes Volumen. Damit können Berechnungen vereinfacht werden. Es kann auch mit einem Elektronenstrahl benutzt werden, um die Oberflächendosis zu erhöhen oder die Eindringtiefe des Strahls zu verringern.

b) Compression Cones: Es ist eine Kegel mit geschlossenem Ende, der gegen die Haut des Patienten gedrückt wird, dort wo der Strahl einfällt. Der Druck zwingt im Prinzip das Gewebe in einer Luftspalte unter dem Kegel und bietet eine flache Oberfläche für den eintretenden Strahl.

c) Tissue-Compensating Files: Das Problem bei Bolus ist, dass bei hohen Photonenenergien, der Dosisaufbau eher im Bolus stattfindet und nicht im oberflächlichen Gewebe des Patienten. Die Oberflächendosis steigt also an. Der hautschonende Vorteil ist eher reduziert oder verloren. Statt Bolus, werden Tissue-Compensating Files verwendet, um den Strahleneingang der Oberfläche abzuflachen. Diese bestehen aus Aluminium, Bronze, Wachs, Blei, etc. Sie werden bei einer Distanz über die Haut des Patienten platziert. Also der Filter wird in Abstand zur Oberfläche angebracht.
