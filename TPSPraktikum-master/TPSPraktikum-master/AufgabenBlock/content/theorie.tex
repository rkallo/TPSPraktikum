\section{Theorie Aufgaben}
\label{sec:Theorie}

\textbf{1. Erläutern Sie drei Bestrahlungsparameter durch deren Einfluss $\si{\percent}$$D_n$ ansteigt.}


$\si{\percent}$$D_n$ variiert mit der Tiefe, Feldgröße, SSD und Energie des Strahls.
Zu den drei Bestrahlungsparameter gehören die Erhöhung der Feldgröße, die Vergrößerung der SSD und die Vergrößerung der Photonenenergie.
Bei der Photonenenergie steigt $\si{\percent}$$D_n$ zunächst schnell unter der Oberfläche an, bis die Tiefe der Maximaldosis erreicht ist. Hinter diese Tiefe nimmt die Dosis wieder ab. Das hängt also mit dem Aufbaueffekt ab. 

\textbf{
2. Erläutern Sie die Begriffe TAR (Tissue-air ratio) und BSF (Backscattering factor) und deren Abhängigkeit vom SSD.}

TAR ist definiert als: 
\begin{equation*}
\text{TAR} = \frac{D_d}{D_\text{air}}
\end{equation*}
wobei $D_d$ die absorbierte Dosis an einem Ort im Medium und $D_\text{air}$ die absorbierte Dosis für kleine Gewebemasse, die an derselben Stelle in Luft umgeben ist. Manchmal wird es auch als Dosis im freien Raum $D_\text{fs}$ definiert. Normalerweise wird TAR auf der Zentralachse bestimmt. Außerdem ist TAR unabhängig vom SSD, weil die beiden Dosis $D_d$ und $D_\text{air}$ im selben Abstand von der Quelle gemessen werden. 

BSF ist definiert als: 
\begin{equation*}
\text{BSF} = \frac{D_0}{D_\text{air}}
\end{equation*}
wobei $D_0$ = $D_d$ ist. Es ist im Prinzip TAR, aber bei einer Tiefe in der Dosismaximum $d_m$. Die Photonenstreuung, die in der Tiefe der Maximumdosis $d_m$ stattfindet, ist abhängig von der Menge des darunterliegenden Mediums, Größe, Form und Qualität der Gamma- oder Röntgenstrahlung. Genauso wie TAR, ist BSF auch unabhängig vom SSD aus demselben Grund wie TAR, dass die Dosis $D_d$ und $D_\text{air}$ im selben Abstand von der Quelle gemessen werden. 

\textbf{3. Definieren Sie den Begriff TPR (Tissue-Phantom Ratio).}

Es wird definiert als das Verhältnis der Dosis bei einer gegebenen Tiefe in einem Phantom zur Dosis im Referenzpunkt. Das ist üblicherweise $\SI{10}{\centi\meter}$.

\textbf{4. Wie können Tiefendosiskurven, die für die Bestrahlungsplanung von nöten sind, ermittelt werden? Warum werden diese benötigt?}

Zur Messung von Tiefendosiskurven wird in der Regel ein Wasserphantom benutzt. Die Messungen erfolgen in der Regel auch mit einer Ionisationskammer, da Wasser in Bezug auf Strahlungsstreuung und -absorption dem menschlichen Gewebe sehr ähnlich ist. Diese werden zur Kalibrierung des Dosiswertes im Beschleuniger als auch zur Prüfung und Kalibrierung des Therapie Plansystems benötigt. Diese werden ermittelt, indem die relative Dosis entlang der Bestrahlungachse im Gewebe angegeben wird.

\textbf{5. Nennen Sie verschiedene Gegenstände (mindestens 3!) mit denen Ungleichmäßigen in der Hautoberfläche ausgeglichen werden können.}

a) Bolus: Es produziert eine nah flache Oberfläche über ein fast homogenes Volumen. Damit können Berechnungen vereinfacht werden. Es kann auch mit einem Elektronenstrahl benutzt werden, um die Oberflächendosis zu erhöhen oder die Eindringtiefe des Strahls zu verringern.

b) Compression Cones: Es ist eine Kegel mit geschlossenem Ende, der gegen die Haut des Patienten gedrückt wird, dort wo der Strahl einfällt. Der Druck zwingt im Prinzip das Gewebe in einer Luftspalte unter dem Kegel und bietet eine flache Oberfläche für den eintretenden Strahl.

c) Tissue-Compensating Files: Das Problem bei Bolus ist, dass die hohe Energie durch die Röntgen- und Gammastrahlen, den Dosisaufgaben eher im Bolus platziert, als im oberflächlichen Gewebe des Patienten. Die Oberflächendosis steigt also an. Der hautschonende Vorteil ist eher reduziert oder verloren. Statt Bolus, werden Tissue-Compensating Files verwendet, um den Strahleneingang der Oberfläche abzuflachen. Diese bestehen aus Aluminium, Bronze, Wachs, Blei, etc. Sie werden bei einer Distanz über die Haut des Patienten platziert. Also der Filter wird in Abstand zur Oberfläche angebracht.



 


