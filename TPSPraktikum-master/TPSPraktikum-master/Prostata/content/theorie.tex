\section{Einleitung}
\label{sec:Einleitung}
Die häufigste Krebserkrankung bei Männern ist das Prostatakarzinom.
Diese wird häufig erst spät entdeckt, da zu Beginn der Erkrankung keine Symptome gibt.
Nach erfolgreicher Behandlung dieser Erkrankung kann es immer passieren, dass das Prostatakarzinom
erneut auftritt, weshalb regelmäßig der PSA-Wert (Prostata-spezifisches Antigen) regelmäßig im Blut kontrolliert wird. \cite{Prostata} \\
In diesem Fall es dazu gekommen, dass das Prostatakarzinom wieder aufgetreten ist.
Deshalb soll nun der Tumor mit einer kurativ intendierten, perkutan fraktionerten Salvage Radiotherapie behandelt werden.
Eine Salvage Radiotherapie wird durchgeführt, wenn die Erstbehandlung bei dem bösartigen
Tumor nicht erfolgreich war oder der Tumor wiederaufgetreten ist \cite{Salvage}.
