\section{Patientenvorstellung}
\label{sec:Patientenvorstellung}
Bei dem Patienten ist bereits im Jahr 2012 ein Prostatakarzinom diagnostiziert worden.
Da in 2013 ein PSA Anstieg beobachtet wurde, ist eine Prostatastanzbiopsie durchgeführt worden und daraufhin eine Prostatektomie.
Dabei ist die Prostata vollständig entfernt worden und auch die rechtsseitigen Lymphknoten.
Im Jahr 2017 ist ein erneuter PSA Anstieg festgestellt worden, weshalb nun eine Salvage Radiotherapie durchgeführt werden soll.
Der Patient wurde über mögliche Wirkungen und Nebenwirkungen der Strahlentherapie durch den Arzt aufgeklärt.
Zur weiteren Diagnosen gehören Diabetes mellitus Typ II, arterielle
Hypertonie, Herzinsuffizienz, Schlafapnoesyndrom, Hyperlipidämie und bei ihm ist ein Stent an der A. vertebralis vorhanden.
Der Patient ist $\SI{169}{\centi\meter}$ groß, wiegt $\SI{66}{\kilogram}$ und hat berichtet,
dass er etwa $\SI{5}{\kilogram}$ verloren hat. Die Prostata soll mit der Shrinking-Field-Technik bestrahlt werden, wobei
zwei Bestrahlungsserien durchgeführt werden. In der ersten Bestrahlungsserie wird eine Gesamtdosis von $\SI{59,4}{\gray}$ appliziert,
welche in Fraktionen von $\SI{1,8}{\gray}$ in 5 Sitzungen pro Woche bestrahlt werden soll. Bei dieser Sitzung befindet sich im PTV die Prostataloge
und ein Großteil der Samenblase mit einem Sicherheitssaum von $\SI{5}{\milli\meter}$.
In der zweiten Bestrahlungsserie wird mit einer
Gesamtdosis von $\SI{10,8}{\gray}$ ein kleineres PTV bestrahlt. Hierbei sollen 5 Bestrahlungssitzungen mit jeweils $\SI{1,8}{\gray}$ pro Woche stattfinden.
In der zweiten Serie befindet sich auch die Prostataloge und die Samenblasen in dem PTV, allerdings mit einem geringeren Sicherheitssaum.
Damit wird bei der gesamten Therapie eine Dosis von $\SI{70,2}{\gray}$ appliziert.
Es soll erreicht werden, dass die beiden PTVs sicher von der $\SI{95}{\percent}$ Isodosenlinie umschlossen werden.
