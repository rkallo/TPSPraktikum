\section{Patientenvorstellung}
\label{sec:Patientenvorstellung}
Bei dem Patient ist bereits im Jahr 2012 ein Prostatakarzinom diagnostiziert worden. Im Laufe der Jahren wurden vor allem einen PSA Anstieg beobachtet, eine Prostatastanzbiopsie als auch eine Prostatektomie durchgeführt. Anschließend wurde im Jahr 2017 fand eine Konferenz über die Durchführung einer Salvage-Radiotherapie statt, welche im Anschluss für den Patient angewandt werden soll. Der Patient wurde über mögliche Wirkungen und Nebenwirkungen der Strahlentherapie durch den Arzt aufgeklärt. Zur weiteren Diagnosen gehören Diabetes mellitus Typ II, d.h. der Patient ist insulinpflichtig. Außerdem leider dieser an einer arteriellen Hypertonie, hat eine Herzinsuffizienz und bei ihm ist ein Stent an der A. vertebralis vorhanden. Hinzu kommt noch das Schlafapnoesyndrom und eine Hyperlipidämie. Der Patient ist $\SI{169}{\centi\meter}$ groß, wiegt $\SI{66}{\kilogram}$ und hat berichtet, dass er etwa $\SI{5}{\kilogram}$ verloren hat. Die Prostata soll mit der Shrinking-Field-Technik bestrahlt werden, d.h. es soll zwei Bestrahlungsserien durchgeführt werden. In der ersten Bestrahlungsserie wird eine Gesamtdosis von $\SI{59,4}{\gray}$ appliziert, welche in Fraktionen von $\SI{1,8}{\gray}$ in 5 Sitzungen pro Woche appliziert werden soll. Die Prostataloge wird unter Einschluss der Samenblasenloge mit einem Sicherheitssaum von $\SI{5}{\milli\meter}$ bestrahlt. In der zweiten Bestrahlungsserie wird mit einer Gesamtdosis von $\SI{10,8}{\gray}$ ein kleineres PTV bestrahlt. Hierbei sollen 5 Bestrahlungssitzungen a $\SI{1,8}{\gray}$ pro Woche stattfinden. Im Gegensatz zur ersten Bestrahlungsserie findet die zweite Serie unter Einschluss der Samenblasen mit einem geringeren Sicherheitssaum statt. Damit wird bei der gesamten Therapie eine Dosis von $\SI{70,2}{\gray}$ appliziert. Es soll erreicht werden, dass die beiden PTVs sicher von der $\SI{95}{\percent}$ Isodosenlinie umschlossen werden und dass es auf Target Mean dosiert wird.